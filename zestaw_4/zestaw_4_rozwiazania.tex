\documentclass[a4paper,12pt]{article}
\usepackage[utf8]{inputenc}
\usepackage[polish]{babel}
\usepackage{amsmath,amssymb}
\usepackage{geometry}
\geometry{a4paper, margin=1in}
\usepackage[T1]{fontenc}

\title{TIFS R - Ćwiczenia Tydzień 4}
\author{}
\date{11 lutego 2026}

\begin{document}

\maketitle

\section*{I. Piekarz, Liouville, Boltzmann}

\subsection*{Zadanie 1}
Rozpatrzmy przekształcenie piekarskie (na wykładzie było podobne przekształcenie kota Arnolda) - najpierw jednostkowy kwadrat jest rozciągany w kierunku $x$ o czynnik $2$ przy jednoczesnym kurczeniu wzdłuż kierunku $y$ o taki sam czynnik. Następnie układ taki jest rozcinany na pół i prawa część kładziona jest na górę lewej. Kroki te w zamyśle mają naśladować zagniatanie ciasta przez piekarza, stąd nazwa. Matematycznie, przekształcenie to możemy zapisać jako:
\begin{equation}
(x', y') = B(x, y) = \begin{cases} (2x, y/2), & x < 1/2 \\ (2x - 1, (y+1)/2), & x \geq 1/2 \end{cases}
\end{equation}

Czy przekształcenie to jest odwracalne? Jeśli tak, to znajdź przekształcenie odwrotne.

\textbf{Rozwiązanie:}

Jest odwracalne, a przekształcenie odwrotne to:
\begin{equation}
(x', y') = B^{-1}(x, y) = \begin{cases} (x/2, 2y), & y < 1/2 \\ ((x+1)/2, 2y - 1), & y \geq 1/2 \end{cases}
\end{equation}

\subsection*{Zadanie 2}
Czy przekształcenie piekarskie zachowuje pole w przestrzeni fazowej?

\textbf{Rozwiązanie:}

Tak, wystarczy rozpatrzyć jakobian i pokazać, że jest równy $1$.

\subsection*{Zadanie 3}
Niech $\rho(x, y)$ opisuje gęstość prawdopodobieństwa znalezienia układu w danym punkcie $(x, y)$ przestrzeni fazowej. Znaleźć równanie Liouville'a rządzące ewolucją $\rho$.

\textbf{Rozwiązanie:}

Równanie Liouville'a dostaniemy wychodząc z:
\begin{equation}
\rho_n(\Gamma) = \int d\Gamma' \rho_{n-1}(\Gamma')\delta(\Gamma - B(\Gamma'))
\end{equation}

Co daje:
\begin{equation}
\rho_n(x, y) = \rho_{n-1}(B^{-1}(x), B^{-1}(y))
\end{equation}

\subsection*{Zadanie 4}
Zdefiniujmy też zredukowaną funkcję rozkładu, która mierzy gęstość prawdopodobieństwa znalezienia układu w punkcie przestrzeni fazowej o danej wartości współrzędnej $x$ (bez względu na wartość $y$):
\begin{equation}
W(x) = \int_0^1 \rho(x, y)dy
\end{equation}

Pokaż, że $W_n(x, y)$ spełnia następujące dyskretne równanie Boltzmanna:
\begin{equation}
W_n(x) = \frac{1}{2}\left[W_{n-1}\left(\frac{x}{2}\right) + W_{n-1}\left(\frac{x+1}{2}\right)\right]
\end{equation}

Jaki jest rozkład równowagowy $W_{\text{eq}}(x)$? Pokaż następnie, że entropia Gibbsa związana z rozkładem $W(x)$, $S_G = -k\int_0^1 W(x)\log W(x)dx$ nie maleje podczas kolejnych iteracji przekształcenia piekarskiego.

\textbf{Rozwiązanie:} Rozwiązanie - u Dorfmana, strony 85-86. Warto pokazać też rozwiązanie równania na $W_n$ (7.8-7.10).

\section*{III. Ergodyczność}

Przećwiczmy też trochę pojęcie ergodyczności. Na wykładzie pojawiło się ono w formie równości pomiędzy średnimi czasowymi:
\begin{equation}
\langle f(\Gamma_0)\rangle_t = \lim_{T \to \infty} \frac{1}{T}\int_0^T f(\Gamma(\Gamma_0, t))dt
\end{equation}
oraz średnimi po przestrzeni fazowej:
\begin{equation}
\langle f \rangle_V = \frac{1}{\mu(\Omega)}\int_\Omega f(\Gamma)d\Gamma
\end{equation}
gdzie $f$ jest dowolną funkcją całkowalną względem miary $d\Gamma$, a $\Omega$ to cała przestrzeń, w której zachodzi ruch. Dla układu ergodycznego te średnie są równe:
\begin{equation}
\langle f(\Gamma_0)\rangle_t = \langle f \rangle_V
\end{equation}
dla prawie wszystkich punktów $\Gamma_0$. W szczególności, jeśli za $f$ weźmiemy funkcję charakterystyczną jakiegoś regionu przestrzeni fazowej $\Omega$, to dostaniemy popularne sformułowanie, że trajektoria spędza równe czasy w obszarach o równej objętości w przestrzeni fazowej.

\subsection*{Zadanie 5}
Na rozruszanie: pokaż, że ruch pojedynczej cząstki w bilardzie kołowym nie jest ergodyczny.

\textbf{Rozwiązanie:} Patrz poniższy rysunek: kąt padania równy kątowi odbicia, wszystkie kąty centralne ($\alpha$) takie same.

\end{document}