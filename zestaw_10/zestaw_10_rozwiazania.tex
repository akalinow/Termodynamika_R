\documentclass[a4paper,12pt]{article}
\usepackage[utf8]{inputenc}
\usepackage[polish]{babel}
\usepackage{amsmath,amssymb}
\usepackage{geometry}
\usepackage[T1]{fontenc}
\geometry{a4paper, margin=1in}

\title{TIFS R - Ćwiczenia Tydzień 10}
\author{}
\date{11 lutego 2026}

\begin{document}

\maketitle

\section*{Równowaga faz i przejścia fazowe}

\subsection*{Zadanie 1}
Rozpatrzmy dwa pojemniki z jednoatomowymi gazami doskonałymi $A$ i $B$, które w chwili początkowej były oddzielone przegrodą, a w chwili końcowej - zmieszane. Załóżmy, że zmiennymi niezależnymi (kontrolowanymi przez kontakt z otoczeniem) są ciśnienie $p$ i temperatura $T$. Pokaż, że proces mieszania jest nieodwracalny.

\textbf{Rozwiązanie:}

Odpowiednim potencjałem termodynamicznym jest tu $G = \mu N$. Jeżeli okaże się, że $\Delta G < 0$, to proces będzie nieodwracalny. Ale $G = \mu N$, zaś wzór na potencjał chemiczny gazu doskonałego w zmiennych $p,T$ wyprowadzaliśmy jakiś czas temu i miał postać:
\begin{equation}
\mu_i = kT \log\left(\frac{p}{kT}\left(\frac{h^2}{2\pi m_i kT}\right)^{3/2}\right) = kT \log p + f_i(T)
\end{equation}

Stąd na początku:
\begin{equation}
G_p = G_A + G_B = N_A \mu_A + N_B \mu_B = NkT \log p + N_A f_A(T) + N_B f_B(T)
\end{equation}

Natomiast po zmieszaniu, ciśnienia każdego ze składników zmaleją (względem ciśnienia początkowego), a ciśnienie całkowite - nie zmieni się. Zgodnie z prawem Daltona:
\begin{equation}
p_A = \frac{N_A}{N}p, \quad p_B = \frac{N_B}{N}p
\end{equation}

A zatem końcowy potencjał Gibbsa:
\begin{equation}
G_k = N_A kT \log p_A + N_B kT \log p_B + N_A f_A(T) + N_B f_B(T)
\end{equation}

Różnica potencjałów:
\begin{equation}
\Delta G = N_A kT \log\frac{p_A}{p} + N_B kT \log\frac{p_B}{p}
\end{equation}

Można przekształcić do postaci:
\begin{equation}
\Delta G = N_A kT \log\frac{N_A}{N} + N_B kT \log\frac{N_B}{N}
\end{equation}

Dla małych wartości:
\begin{equation}
\Delta g = kT(x \log x + (1 - x)\log(1 - x))
\end{equation}

dla $x = \frac{N_A}{N}$. Nietrudno zobaczyć, że $\Delta g(x)$ jest symetryczna względem $x = \frac{1}{2}$, poza tym $\Delta g(0) = \Delta g(1) = 0$ oraz pochodna zeruje się tylko dla $x = \frac{1}{2}$, w którym to punkcie funkcja przyjmuje wartość ujemną. A zatem $\Delta g(x) < 0$ i faktycznie proces jest nieodwracalny.

Z drugiej strony pamiętamy, że $G = U - TS + pV$. W tym przypadku $U = \text{const.}$, $T = \text{const.}$ oraz $p = \text{const.}$. Nietrudno też zauważyć, że nie zmieni się objętość zajmowana przez gazy, gdyż przed zmieszaniem $V = V_A + V_B = \frac{N_A kT}{p} + \frac{N_B kT}{p}$, a po zmieszaniu $V = \frac{(N_A + N_B)kT}{p}$. Możemy więc proces izobaryczno-izotermicznego zmieszania wyobrazić sobie jako usuwanie przegrody w naczyniu o ustalonej objętości $V$. Tak czy inaczej w naszym przypadku:
\begin{equation}
\Delta G = -T \Delta S
\end{equation}

a zatem:
\begin{equation}
\Delta S = -k\left(N_A \log\frac{N_A}{N} + N_B \log\frac{N_B}{N}\right) \equiv \Delta S_{\text{mix}}
\end{equation}

\end{document}
