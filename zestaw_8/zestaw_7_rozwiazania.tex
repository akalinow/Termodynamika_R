\documentclass[a4paper,12pt]{article}
\usepackage[utf8]{inputenc}
\usepackage[T1]{fontenc}
\usepackage[polish]{babel}
\usepackage{lmodern} % Ensure proper font rendering for Polish characters
\usepackage{amsmath}
\usepackage{amssymb}
\usepackage{amsfonts}
\usepackage{geometry}
\geometry{a4paper, margin=1in}

\title{TIFS R - Ćwiczenia tydzień 7}
\author{}
\date{}

\begin{document}

\maketitle

\section*{I zasada termodynamiki i cykle termodynamiczne}

Przećwiczmy I zasadę termodynamiki, pojęcie ciepła, pracy, procesu odwracalnego, licząc kilka cykli termodynamicznych. Ciałem roboczym będzie za każdym razem gaz doskonały. Dobrze jest przypomnieć, że dla tego gazu $u = c_v T$, gdzie $c_v = \frac{3}{2} k$, a zatem energia wewnętrzna zależy tylko od temperatury, jak również przypomnieć równanie stanu. Uwaga na dwa sposoby zapisu tych wielkości (na cząstkę i na mol - ze stałą gazową).

Wszystkie poniższe procesy są odwracalne (kwazistatyczne).

\subsection*{Zadanie 1}

Znajdź pracę wykonaną nad gazem doskonałym i ilość ciepła zaabsorbowanego przez ten gaz podczas procesu cyklicznego $(p_1, V_1) \to (p_1, V_2) \to (p_2, V_2) \to (p_2, V_1) \to (p_1, V_1)$.

\subsection*{Rozwiązanie}

Wracamy do punktu wyjścia, więc $W = -Q$ (bo $\Delta U = 0$); podczas przemiany izochorycznej $W = 0$, podczas izobarycznej $W = -p\Delta V$, stąd całkowita praca $W = (V_2 - V_1)(p_2 - p_1)$. Interpretacja - pole prostokąta.

\subsection*{Zadanie 2}

Zdefiniować ciepła właściwe przy stałej objętości i ciśnieniu i udowodnić, że dla gazu doskonałego $C_p = C_v + R$.

\subsection*{Rozwiązanie}

Jest dużo dowodów tej równości (patrz np. Kubo - Zad 3 str.20), ja bym jednak chciała, żebyśmy udowodnili ją wykorzystując rozprężanie do próżni (tzw. cykl Mayera) - Kubo zad 11 str. 33). Podkreślcie koniecznie, że to proces nierównowagowy i pojęcie ciśnienia gazu nie ma tam sensu (patrz dyskusja w Kubo).

\subsection*{Zadanie 3}

Równanie adiabaty dla gazu doskonałego: Pokazać, że podczas procesu adiabatycznego (kwazistatycznego) spełniony jest następujący związek między ciśnieniem a objętością:
\[ pV^\gamma = \text{const.} \]

Znaleźć pracę gazu nad otoczeniem podczas kwazistatycznego rozszerzania adiabatycznego z $(p_1, V_1, T_1)$ do $(p_2, V_2, T_2)$. (Kubo - Zad 6 str. 22)

\subsection*{Zadanie 4}

Cykl Carnota dla gazu doskonałego (izoterma-adiabata-izoterma-adiabata). Pokazać równanie Clausiusa:
\[ \frac{Q_1}{T_1} + \frac{Q_2}{T_2} = 0 \]

(patrz Kubo 9/32, rozwiązanie tamże)

\subsection*{Zadanie 5}

Rozpatrzmy gaz fotonów, dla którego $p = \frac{1}{3} u$, $u = \sigma T^4$. Obliczyć entropię tego gazu, wyprowadzić równanie adiabaty, oraz przeanalizować cykl Carnota z udziałem tego gazu. (Kubo 20/36 i załączony artykuł).

\end{document}
