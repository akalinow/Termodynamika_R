\documentclass[a4paper,12pt]{article}
\usepackage[utf8]{inputenc}
\usepackage[polish]{babel}
\usepackage{amsmath,amssymb}
\usepackage{geometry}
\geometry{a4paper, margin=1in}
\usepackage[T1]{fontenc}

\title{TIFS R - Ćwiczenia Tydzień 8}
\author{}
\date{11 lutego 2026}

\begin{document}

\maketitle

\section*{Silniki, lodówki, entropia}

Silnik cieplny - urządzenie do przetwarzania energii cieplnej na energię mechaniczną; silnik cieplny pobiera ciepło ze źródła o temperaturze wyższej (ciepło spalania paliwa w silniku spalinowym lub w turbinie), przetwarza jego część na pracę mechaniczną, a resztę oddaje w chłodnicy.

Wynalazcą pierwszego urządzenia zmieniającego ciepło na pracę był Heron z Aleksandrii (I w. n.e.). Skonstruował on mianowicie tzw. banię Herona obracaną strumieniami pary wydostającymi się z niej przez odpowiednio zagięte rurki - wzór turbiny parowej.

Sprawność silnika - stosunek pracy wykonanej przez układ do pobranego ciepła:
\begin{equation}
\eta = \frac{W}{Q_1}
\end{equation}

\subsection*{Zadanie 1}
Udowodnić równość Clausiusa dla cyklu Carnota, a następnie znaleźć sprawność silnika Carnota.

\textbf{Rozwiązanie:}

Wzór Clausiusa wyszedł nam już dwukrotnie - dla gazu doskonałego i gazu fotonów, teraz dostaniemy sformułowanie ogólne i okaże się, że tamte wyniki były nieprzypadkowe. Rozwiązanie jest proste, ale subtelne - omawiajcie każdy krok i podkreślajcie, z której zasady kiedy korzystamy.

Najpierw z I-szej zasady termodynamiki:
\begin{equation}
0 = \Delta U_u = Q_1 + Q_2 + W
\end{equation}

Wielkości z indeksem $u$ odnoszą się do układu, z indeksem $1$ - grzejnik, $2$ - chłodnica. Potem z II-ej zasady (proces odwracalny!):
\begin{equation}
0 = \Delta S = \Delta S_1 + \Delta S_2 + \Delta S_u
\end{equation}

Układ wraca do stanu początkowego, więc $\Delta S_u = 0$, stąd ostatecznie:
\begin{equation}
\Delta S_1 + \Delta S_2 = 0
\end{equation}

Zmiany entropii termostatów policzyć łatwo, z definicji mają bowiem stałe temperatury, więc:
\begin{equation}
Q_1 = -T_1 \Delta S_1
\end{equation}

Uwaga na znak: $T_1 \Delta S_1$ to ciepło oddane przez układ termostatowi. Analogicznie:
\begin{equation}
Q_2 = -T_2 \Delta S_2
\end{equation}

Skąd dostajemy równość Clausiusa:
\begin{equation}
\frac{Q_1}{T_1} + \frac{Q_2}{T_2} = 0
\end{equation}

Pozwala to też na obliczenie sprawności:
\begin{equation}
\eta = \frac{-W}{Q_1} = \frac{Q_1 + Q_2}{Q_1} = 1 - \frac{T_2}{T_1}
\end{equation}

Typowe wielkości - w elektrociepłowniach termostaty mają ok. $700\,\mathrm{K}$, chłodnice (woda) - $300\,\mathrm{K}$, co dawałoby idealne sprawności ok. $57\%$, w rzeczywistości silniki nie są idealne i jest gorzej - ok. $33\%$. (Dla porównania elektrownie niecieplne - np. wodne mają sprawność zamiany energii potencjalnej wody na prąd elektryczny rzędu $85\%$).

Uwaga: Można również odwrócić kota ogonem i przyjąć stwierdzenie: \textit{Stosunek $|Q_2|/|Q_1| (T_2/T_1)$ jest taki sam dla wszystkich procesów Carnota (niezależnie od substancji roboczej) operujących między termostatami o temperaturze $T_1$ i $T_2$} za alternatywne sformułowanie drugiej zasady termodynamiki.

\subsection*{Zadanie 2}
Pokazać, że żaden inny silnik cieplny nie będzie działał sprawniej od silnika Carnota.

\textbf{Rozwiązanie:}

(Przytaczam rozwiązanie za skryptem G.K do wykładu M.N - konwencja oznaczeń jest troszkę inna niż zwykle).

Niech w czasie jednego cyklu silnik wykonuje pracę $W$, przy czym w trakcie cyklu pobiera z otoczenia ciepło $Q_+$ oraz oddaje do otoczenia ciepło $Q_-$. W czasie cyklu zmiana energii wewnętrznej silnika wynosi zero, więc z pierwszej zasady termodynamiki uzyskujemy:
\begin{equation}
0 = \Delta U = Q_+ - (W + Q_-)
\end{equation}

czyli:
\begin{equation}
W = Q_+ - Q_-
\end{equation}

Zgodnie z nierównością Clausiusa:
\begin{equation}
\oint \frac{\delta Q}{T} \leq 0
\end{equation}

Rozdzielmy cykl na odcinki odpowiadające poszczególnym kierunkom przepływu ciepła między silnikiem a otoczeniem:
\begin{equation}
\int_+ \frac{\delta Q_+}{T} - \int_- \frac{\delta Q_-}{T} \leq 0
\end{equation}

Temperaturę w pierwszej całce oszacujmy przez najwyższą temperaturę osiąganą w trakcie cyklu, zaś w drugiej - najniższą. Wówczas...

\end{document}
