\documentclass[a4paper,12pt]{article}
\usepackage[utf8]{inputenc}
\usepackage[T1]{fontenc}
\usepackage[polish]{babel}
\usepackage{lmodern} % Ensure proper font rendering for Polish characters
\usepackage{amsmath}
\usepackage{amssymb}
\usepackage{amsfonts}
\usepackage{geometry}
\geometry{a4paper, margin=1in}

\title{TIFS R - Ćwiczenia tydzień 11}
\author{}
\date{}

\begin{document}

\maketitle

\section*{Oscylatory, oscylatory, oscylatory...}

Zadanie o układzie $N$ niezależnych trójwymiarowych kwantowych oscylatorów harmonicznych rozwiązywaliśmy przy okazji rozkładu mikrokanonicznego. Teraz zróbmy to szybko z kanonicznego.

\subsection*{Zadanie 1}

Korzystając z rozkładu kanonicznego znajdź energię swobodną i energię wewnętrzną układu $N$ niezależnych trójwymiarowych kwantowych oscylatorów harmonicznych.

\subsection*{Rozwiązanie}

Z niezależności:
\[
Z(N, V, T) = Z_1(V, T)^{3N}
\]
Gdzie $Z_1$ jest pojedynczym jednowymiarowym oscylatorem. Jego suma statystyczna jest równa:
\[
Z_1 = \sum_{n=0}^\infty e^{-\beta \hbar \omega \left(n + \frac{1}{2}\right)} = \frac{e^{-\frac{\beta \hbar \omega}{2}}}{1 - e^{-\beta \hbar \omega}} = \frac{1}{2 \sinh \frac{\beta \hbar \omega}{2}}
\]

Zatem:
\[
Z(N, V, T) = \left(\frac{1}{2 \sinh \frac{\beta \hbar \omega}{2}}\right)^{3N}
\]

Energia swobodna:
\[
F = 3N \left(\frac{1}{2} \hbar \omega + kT \ln \left(1 - e^{-\beta \hbar \omega}\right)\right)
\]

Energia wewnętrzna:
\[
U = -\frac{\partial}{\partial \beta} \ln Z = 3N \left(\frac{1}{2} \hbar \omega + \frac{\hbar \omega}{e^{\beta \hbar \omega} - 1}\right)
\]

\end{document}
