\documentclass[11pt,a4paper]{article}

\usepackage[polish]{babel}
\usepackage[utf8]{inputenc}
\usepackage{polski}
\usepackage[T1]{fontenc}
\usepackage{indentfirst}
\usepackage{wrapfig}    % for wrapping figures, tables
\usepackage{isotope}
\usepackage{hyperref}
\usepackage{amsmath}
\usepackage{amssymb}
\usepackage{bm}
\usepackage{gensymb}
\usepackage{epsfig}
\usepackage{graphics}
\usepackage[shortlabels]{enumitem}
\usepackage{xspace}
\xspaceaddexceptions{[]\{\}}

%
%
%fixpagesize
\pagestyle{empty}
\addtolength{\textwidth}{4cm}
\addtolength{\textheight}{6cm}
\addtolength{\evensidemargin}{-3cm}
\addtolength{\oddsidemargin}{-2cm}
\addtolength{\topmargin}{-3cm}
\parindent=0cm

%
%
% Changes figure placing algorithm
\renewcommand{\topfraction}{1}       % maximal fraction of a page allowed for figures
\renewcommand{\textfraction}{0.15}   % minimal number of text for figure-text shared pages
\renewcommand{\floatpagefraction}{0.95} % if two above does not help, this could do the job 
                                        % must be: floatpagefraction < topfraction !!!!
\renewcommand{\textfraction}{0} % minimum fraction of page, which must be  devoted to text
\renewcommand{\topfraction}{1}  % maximum fraction at top, which can be occupied whit floats
\setcounter{totalnumber}{400}   % increase the number of floats for one page
\setcounter{topnumber}{200}     % at all/top/bottom.
\setcounter{bottomnumber}{200}  %

%
%
%small distance in list/item/enum for enumitem package
\setlist[itemize,enumerate]{topsep=0em}
\setlist{noitemsep}

%Nuclear notations: usage \nucl{235}{92}{U}. Math mode optional
\newcommand{\nucl}[3]{\ensuremath{
  \phantom{\ensuremath{^{#1}_{#2}}}
  \llap{\ensuremath{^{#1}}}
  \llap{\ensuremath{_{\rule{0pt}{.75em}#2}}}
  \mbox{#3} } 
} 

%print zadanie #
\newcounter{zadanie}\newcommand{\zadanie}[1][]{\addtocounter{zadanie}{1} ~\\  {\bf \emph{Zadanie \arabic{zadanie} #1 }} \\}
\newcounter{zaddom}\newcommand{\zaddom}[1][]{\addtocounter{zaddom}{1} ~\\  {\bf \emph{Zadanie domowe \arabic{zaddom} #1 }} \\}

%dynamic solutions hiding
\usepackage{comment}

\newif\ifshowsolutions

%\showsolutionstrue   % ← show solutions
% \showsolutionsfalse % ← hide solutions

\def\showsolutions{}

\newif\ifshowsolutions
\ifdefined\showsolutions
  \showsolutionstrue
\else
  \showsolutionsfalse
\fi

\ifshowsolutions
  \newenvironment{solution}{\vspace{1cm} \par\textbf{Rozwiązanie.} \vspace{1cm}}{}
\else
  \excludecomment{solution}
\fi
%%%%%%%%%%%%%%%%%%%%%%%%%%%%%%%%%%%%%%%%%%
\begin{document}   

%%%%%%%%%%%%%%%%%%%%%%%%%%%%%%%%%%%%%%%%%%%%%%%%%%%%%
\begin{centering}
  {\bf {\Large
  Termodynamika i Fizyka Statystyczna R 
  Zestaw III: Rachunek prawdopodobieństwa - kontynuacja
  }} \\  
\end{centering}

\vspace{1cm}
%%%%%%%%%%%%%%%%%%%%%%%%%%%%%%%%%%%%%%%%%%%%%%%%%%%%%%%

\zadanie[Model dyfuzji Bernoulliego-Laplace'a]

Rozważmy model mieszania dwu, nieściśliwych, płynów. 
Płyny są reprezentowane przez dwie urny:  A, B, w których znajduje się
$_A$ i n$_B$ kul. Kule mają dwa kolory: czerwony i czarny.
Kul czerwonych jest n$_{r}$, a kul czarnych n$_{c}$ = n$_{A}$ + n$_{B}$ - n$_{r}$.
W kolejnych chwilach czasu wybieramy po jednej kuli z każdej urny i zamieniamy je miejscami.
Proszę:
\begin{enumerate}
  \item oszacować równowagową wartość oczekiwaną ułamka 
        kul czerwonych w urnach A i B: $\langle n_{r}/n_A \rangle_{eq}$,
        $\langle n_{r}/n_B \rangle_{eq}$
  \item znaleźć stacjonarny rozkład prawdopodobieństwa 
        liczby kul czerwonych w urnie A: $P^{eq}(n_{r})$,
   \item znaleźć czas powrotu do pewnego stanu historycznego
\end{enumerate}

\begin{solution}

\newpage  
\end{solution}
%%%%%%%%%%%%%%%%%%%%%%%%%%%%%%%%%%%%%%%%%%%%%%%%%%%%%%%%%%%%
%%%%%%%%%%%%%%%%%%%%%%%%%%%%%%%%%%%%%%%%%%%%%%%%%%%%%%%%%%%%
\zadanie[Psy Ehrenfestów]


W modelu psów Ehrenfestów rozważa się dwa psy, A i B, na których łącznie 
siedzi N pcheł. W kolejnych chwilach czasu losowo wybrana pchła przeskakuje 
z psa na psa. Model ten miał w założeniu opisywać dynamikę
mieszania dwóch gazów w pojemnikach A i B o równej objętości.
Proszę:
\vspace{0.5cm}
\begin{enumerate}
  \item wypisać warunkowe prawdopodobieństwa przejść z jednego stanu do drugiego:
  \begin{align*}
    p_{i,j} = P(n_{A}(t+1) = i | n_{A}(t) = j)
  \end{align*}
  \item Rozważ modyfikację tego modelu, w której w pojemniku A znajduje się demon, 
  który - z prawdopodobieństwem $\lambda$ - próbuje powstrzymać cząstkę przed 
  przeskoczeniem do pojemnika B. Demon nie zmienia prawdopodobieństw skoków z B do A. 
  W rezultacie
$p^{\lambda}_{k-1,k} = (1 - \lambda) p^0_{k-1,k}$;
$p^{\lambda}_{k,k} = \lambda p^0_{k-1,k}$;
$p^{\lambda}_{j,k} = p^0_{j,k}$ dla $j \neq k, k-1$
gdzie $p^0_{j,k}$ są standardowymi prawdopodobieństwami przejścia 
(bez demona, dla $\lambda = 0$). Zauważmy, że w
wyniku działania demona pojawia się niezerowa wartość $p^{\lambda}_{k,k}$.

Dla modelu z demonem wypisz równanie na ewolucję rozkładu $P_i(t)$, oblicz 
rozkład równowagowy $P_i^{eq}$ oraz $\langle N_A \rangle^{eq}$ - średnią 
liczbę cząstek w pojemniku A w stanie równowagi. Zinterpretuj otrzymane wyniki.

{\bf Wskazówka:} Możesz (ale nie musisz) skorzystać z warunku równowagi szczegółowej


  \item znajdź ciągłą wersję równania na ewolucję rozkładu prawdopodobieństwa, 
  interpretując nadmiarową liczbę cząstek w pojemniku A: $k = N_A - N/2$
  jako położenie (x) pewnej efektywnej cząstki Browna: $x = k\delta x$ oraz 
  przyjmując, że skoki następują co odstęp czasu $\delta t$ a następnie 
  przechodząc do granicy $\delta x \to 0$, $ \delta t \to 0$, $N \to \infty$ 
  w taki sposób, że: 
  $\frac{\delta x^2}{\delta t} = D$ i 
  $\frac{\delta x}{\delta t} = \Gamma$, gdzie D i $\Gamma$ są pewnymi stałymi.

\end{enumerate}

\begin{solution}


\newpage
\end{solution}
%%%%%%%%%%%%%%%%%%%%%%%%%%%%%%%%%%%%%%%%%%%%%%%%%%%%%%%%%%%%%
%%%%%%%%%%%%%%%%%%%%%%%%%%%%%%%%%%%%%%%%%%%%%%%%%%%%%%%%%%%%%
\zadanie[Prawo wzrostu entropii]

Rozważny układ o n stanach, między którymi zachodzą przejścia 
z prawdopodobieństwami $p_{ij}$ na jednostkę czasu. W stanie 
równowagi wszystkie stany są równoprawdopodobne. 
W stanie początkowym układ znajduje się w stanie nierównowagowym,
w którym prawdopodobieństwa obsadzeń poszczególnych stanów dane są przez $P_i$.
Pokaż, że w takim przypadku z warunku mikroskopowej odwracalności 
(równowagi szczegółowej) dla (jednorodnego) rozkładu równowagowego:
\begin{align*}
  p_{ij} = p_{ji}
\end{align*}
i entropii zdefiniowanej jako:
\begin{align*}
  S = -\sum_i P_i \ln P_i
\end{align*}
wynika, że entropia rośnie z czasem, aż osiągnie wartość maksymalną w stanie równowagi.

\begin{solution}
  
Obliczmy pochodną entropii względem czasu:
\begin{align*}
  \frac{dS}{dt} &= -\sum_i \frac{dP_i}{dt} \ln P_i - \sum_i P_i \frac{1}{P_i} \frac{dP_i}{dt} \\
  &= -\sum_i \frac{dP_i}{dt} \ln P_i - \sum_i \frac{dP_i}{dt}
\end{align*}

Drugi człon sumy jest równy zero, 
ponieważ suma zmian obsadzeń wszystkich stanów musi być równa zero -
liczba cząstek jest stała, więc:
\begin{align*}
  \frac{dS}{dt} &= -\sum_i \frac{dP_i}{dt} \ln P_i
\end{align*}

Prawdopodobieństwo obsadzeń, $P_i$ możemy traktować jako 
liczbę obsadzeń dla układu o stałej liczbie cząstek.
Zmiana liczby obsadzeń jest dana równaniem Master:
\begin{align*}
  \frac{dP_i}{dt} &= \sum_j (p_{ji} P_j - p_{ij} P_i)
\end{align*}

Podstawiając to do wyrażenia na pochodną entropii, otrzymujemy:
\begin{align*}
  \frac{dS}{dt} &= -\sum_i \left( \sum_j (p_{ji} P_j - p_{ij} P_i) \right) \ln P_i \\
  &= -\sum_{i,j} p_{ji} P_j \ln P_i + \sum_{i,j} p_{ij} P_i \ln P_i
\end{align*}

Korzystamy z warunku mikroskopowej odwracalności: $p_{ij} = p_{ji}$
\begin{align*}
  \frac{dS}{dt} &= \sum_{i,j} p_{ij} (P_i - P_j) \ln P_i
\end{align*}

By uprościć to wyrażenie, możemy dodać drugi taki sam człon, 
ale z zamienionymi indeksami i oraz j:
\begin{align*}
  \frac{dS}{dt} &= \frac{1}{2} \sum_{i,j} p_{ij} (P_i - P_j) \ln P_i + \frac{1}{2} \sum_{i,j} p_{ij} (P_j - P_i) \ln P_j \\
  &= \frac{1}{2} \sum_{i,j} p_{ij} (P_i - P_j) (\ln P_i - \ln P_j) = \\
  &= \frac{1}{2} \sum_{i,j} p_{ij} (P_i - P_j) \ln \frac{P_i}{P_j}
\end{align*}

Zbadajmy człony sumy:
\begin{align*}
  (x-y)\ln(x/y) &= (x-y)(\ln x - \ln y) \\
\end{align*}

Możemy rozważyć trzy przypadki:
\begin{itemize}
  \item $x > y$: w tym przypadku $\ln(x/y) > 0$ i $x-y > 0$, więc cały człon jest dodatni
  \item $x < y$: w tym przypadku $\ln(x/y) < 0$ i $x-y < 0$, więc cały człon jest dodatni
  \item $x = y$: w tym przypadku $\ln(x/y) = 0$ i $x-y = 0$, więc cały człon jest równy zero
\end{itemize}

Ostatecznie, dla każdego członu sumy mamy:
\begin{align*}
  (P_i - P_j) \ln \frac{P_i}{P_j} \geq 0 \rightarrow \frac{dS}{dt} \geq 0
\end{align*}

\newpage
\end{solution}
%%%%%%%%%%%%%%%%%%%%%%%%%%%%%%%%%%%%%%%%%%%%%%%%%%%%%%%%%%%%%%
%%%%%%%%%%%%%%%%%%%%%%%%%%%%%%%%%%%%%%%%%%%%%%%%%%%%%%%%%%%%%%
\end{document}