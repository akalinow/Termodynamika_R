\documentclass[T1]{article}
\usepackage[utf8]{inputenc}
\usepackage[polish]{babel}
\usepackage{amsmath}
\usepackage[table,xcdraw]{xcolor}
\usepackage{geometry}
\usepackage{hyperref}
\geometry{a4paper, margin=1in}

\title{TIFS R - Ćwiczenia Tydzień 3}
\author{}
\date{}

\begin{document}

\maketitle

\section*{Model Dyfuzji Bernoulliego-Laplace'a}

Model dyfuzji Bernoulliego-Laplace'a, znany też jako model urnowy Bernoulliego, jest całkiem nieźle opisany pod linkiem:
\begin{center}
\url{https://www.randomservices.org/random/markov/BernoulliLaplace.html#rev1}
\end{center}

\subsection*{Model}
Załóżmy, że mamy dwie urny: $0$ i $1$. Urna $0$ zawiera $j$ kul, a urna $1$ zawiera $k$ kul. Spośród tych $j + k$ kul $r$ jest czerwonych. W każdej chwili czasu wybieramy po jednej kuli z każdej urny i zamieniamy je miejscami. Możemy o tym myśleć jako o modelu mieszania dwóch nieściśliwych płynów.

\textbf{Intuicje:} Czego się spodziewamy? Po długim czasie ułamki czerwonych kul po obu stronach wyrównają się. Jeśli początkowo np. $\frac{x}{k} \gg \frac{r - x}{j}$ (gdzie $x$ to liczba czerwonych kul w urnie $1$), to ułamek kul czerwonych w urnie $1$ będzie szybko malał, a w urnie $0$ - rósł. Po osiągnięciu stanu równowagi zaczną one fluktuować względem $\frac{r}{j + k}$, przy czym fluktuacje te będą tym wyraźniejsze, im mniej jest kul. Załóżmy, że $r \ll j, k$ (mało czerwonych kul - to pozwoli uniknąć różnych przypadków specjalnych).

\end{document}