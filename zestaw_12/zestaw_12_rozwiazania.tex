\documentclass[a4paper,12pt]{article}
\usepackage[utf8]{inputenc}
\usepackage[T1]{fontenc}
\usepackage[polish]{babel}
\usepackage{lmodern} % Ensure proper font rendering for Polish characters
\usepackage{amsmath}
\usepackage{amssymb}
\usepackage{amsfonts}
\usepackage{geometry}
\geometry{a4paper, margin=1in}

\title{TIFS R - Ćwiczenia tydzień 12}
\author{}
\date{}

\begin{document}

\maketitle

\section*{Statystyki kwantowe i własności fermionów}

\subsection*{Zadanie 1}

Dwie nieoddziałujące cząstki znajdują się w studni potencjału o poziomach energetycznych $E_n = n \varepsilon$. Dodatkowo, $n$-ty poziom energetyczny jest zdegenerowany $2n + 1$ razy. Rozważmy osobno przypadek, gdy:
\begin{enumerate}
    \item Cząstki są bozonami o zerowym spinie.
    \item Cząstki są fermionami o spinie $\frac{1}{2}$.
    \item Cząstki podlegają statystyce Boltzmanna o spinie $s$.
\end{enumerate}
Zakładając, że energia nie zależy od wartości spinu:
\begin{enumerate}
    \item[a)] Znajdź liczbę mikrostany odpowiadające danej wartości $E = N \varepsilon$.
    \item[b)] Znajdź kanoniczną sumę statystyczną układu, jeżeli założymy, że jest on w kontakcie z termostatem o temperaturze $T$.
\end{enumerate}

\subsection*{Rozwiązanie}

Rozwiązanie jest w Dalvicie - zad. 4.1. Poświęćcie trochę czasu na omówienie nierówności między odpowiednimi funkcjami rozkładu (punkt c u Dalvita).

\end{document}
