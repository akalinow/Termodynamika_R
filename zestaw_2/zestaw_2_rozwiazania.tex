\documentclass[11pt,a4paper]{article}

\usepackage[polish]{babel}
\usepackage[utf8]{inputenc}
\usepackage{polski}
\usepackage[T1]{fontenc}
\usepackage{indentfirst}
\usepackage{wrapfig}    % for wrapping figures, tables
\usepackage{isotope}
\usepackage{hyperref}
\usepackage{amsmath}
\usepackage{amssymb}
\usepackage{bm}
\usepackage{gensymb}
\usepackage{epsfig}
\usepackage{graphics}
\usepackage[shortlabels]{enumitem}
\usepackage{xspace}
\xspaceaddexceptions{[]\{\}}

%
%
%fixpagesize
\pagestyle{empty}
\addtolength{\textwidth}{4cm}
\addtolength{\textheight}{6cm}
\addtolength{\evensidemargin}{-3cm}
\addtolength{\oddsidemargin}{-2cm}
\addtolength{\topmargin}{-3cm}
\parindent=0cm

%
%
% Changes figure placing algorithm
\renewcommand{\topfraction}{1}       % maximal fraction of a page allowed for figures
\renewcommand{\textfraction}{0.15}   % minimal number of text for figure-text shared pages
\renewcommand{\floatpagefraction}{0.95} % if two above does not help, this could do the job 
                                        % must be: floatpagefraction < topfraction !!!!
\renewcommand{\textfraction}{0} % minimum fraction of page, which must be  devoted to text
\renewcommand{\topfraction}{1}  % maximum fraction at top, which can be occupied whit floats
\setcounter{totalnumber}{400}   % increase the number of floats for one page
\setcounter{topnumber}{200}     % at all/top/bottom.
\setcounter{bottomnumber}{200}  %

%
%
%small distance in list/item/enum for enumitem package
\setlist[itemize,enumerate]{topsep=0em}
\setlist{noitemsep}

%Nuclear notations: usage \nucl{235}{92}{U}. Math mode optional
\newcommand{\nucl}[3]{\ensuremath{
  \phantom{\ensuremath{^{#1}_{#2}}}
  \llap{\ensuremath{^{#1}}}
  \llap{\ensuremath{_{\rule{0pt}{.75em}#2}}}
  \mbox{#3} } 
} 

%print zadanie #
\newcounter{zadanie}\newcommand{\zadanie}[1][]{\addtocounter{zadanie}{1} ~\\  {\bf \emph{Zadanie \arabic{zadanie} #1 }} \\}
\newcounter{zaddom}\newcommand{\zaddom}[1][]{\addtocounter{zaddom}{1} ~\\  {\bf \emph{Zadanie domowe \arabic{zaddom} #1 }} \\}

%dynamic solutions hiding
\usepackage{comment}

\newif\ifshowsolutions

%\showsolutionstrue   % ← show solutions
% \showsolutionsfalse % ← hide solutions

\def\showsolutions{}

\newif\ifshowsolutions
\ifdefined\showsolutions
  \showsolutionstrue
\else
  \showsolutionsfalse
\fi

\ifshowsolutions
  \newenvironment{solution}{\vspace{1cm} \par\textbf{Rozwiązanie.} \vspace{1cm}}{}
\else
  \excludecomment{solution}
\fi
%%%%%%%%%%%%%%%%%%%%%%%%%%%%%%%%%%%%%%%%%%

\begin{document}         

%%%%%%%%%%%%%%%%%%%%%%%%%%%%%%%%%%%%%%%%%%%%%%%%%%%%%
\begin{centering}
  {\bf {\Large
  Termodynamika i Fizyka Statystyczna R 
  Zestaw II: Rachunek prawdopodobieństwa - kontynuacja
  }} \\  
\end{centering}

\vspace{1cm}
%%%%%%%%%%%%%%%%%%%%%%%%%%%%%%%%%%%%%%%%%%%%%%%%%%%%%%%
\zadanie[Jednowymiarowe błądzenie przypadkowe]

Proszę rozważyć jednowymiarowe błądzenie przypadkowe po sieci o kroku $a$ 
i znaleźć:
\begin{itemize}
    \item rozkład prawdopodobieństwa położenia cząstki po $N$ skokach:
          $p(x|N)$, gdzie $x = Na$ 
    \item postać rozkładu w granicy $a \to 0$, $N \to \infty$ i 
          odpowiedniej korelacji między $a$ i $N$.
\end{itemize}

{\bf Wskazówki:} 
\begin{itemize}
\item użyć zmiennej $z = k - N/2$, gdzie $k$ to liczba skoków na prawo, 
\item użyć przybliżenia Stirlinga dla silni
\end{itemize}


\begin{solution}

Położenie to $x = [k - (N - k)]a = (2k - N)a$, gdzie $k$ to liczba skoków na prawo. 
Prawdopodobieństwo znalezienia cząstki w położeniu $x$ po $N$ skokach to
prawdopodobieństwo, sekwencji $k$ skoków na prawo i $N-k$ skoków na lewo,
pomnożone przez liczbę możliwych sekwencji, czyli:
\begin{align*}
    p(x|N) &= \frac{N!}{k!(N-k)!} \cdot \frac{1}{2^N} 
\end{align*}

Zsymetryzujmy wzór, wprowadzając zmienną $z = k - N/2$, 
czyli liczbę skoków na prawo ponad średnią. Wtedy:

\begin{align*}
    p(z) &= \frac{N!}{(N/2 + z)!(N/2 - z)!} \cdot \frac{1}{2^N}
\end{align*}

Przybliżenie Stirlinga dla silni:
\begin{align*}
    \log n! = \frac{1}{2} \log 2\pi + \left(n + \frac{1}{2}\right) \log n - n + \ldots
\end{align*}

Aplikujemy do wzoru na $p(z)$:
\begin{align*}
    \log p \simeq \\
        \frac{1}{2} \log 2\pi + \left(N + \frac{1}{2}\right) \log N - N - \\
       - \left[\frac{1}{2} \log 2\pi + \left(\frac{N}{2} + z + \frac{1}{2}\right) \log \left(\frac{N}{2} + z\right)  - \frac{N}{2} - z\right]  \\
       - \left[\frac{1}{2} \log 2\pi + \left(\frac{N}{2} - z + \frac{1}{2}\right) \log \left(\frac{N}{2} - z\right)  - \frac{N}{2} + z\right]  \\
       - N \log 2 
\end{align*}


Zbierając:

\begin{align*}
    \log p = -\frac{1}{2} \log 2\pi + \left(N + \frac{1}{2}\right) \log N - \left(N + \frac{1}{2}\right) \log \left[\left(\frac{N}{2} + z\right) \left(\frac{N}{2} - z\right) \right] - z \log \frac{N/2 + z}{N/2 - z} - N \log 2
\end{align*}

Wyrażenia typu $\log(\frac{N}{2} + z)$ można przybliżyć dla $N \gg z$ jako:
\begin{align*}
    \log(\frac{N}{2} + z) = \log \frac{N}{2} + \log(1 + \frac{2z}{N}) \approx \log \frac{N}{2} + \frac{2z}{N}
\end{align*}


Używając tego przybliżenia, dostajemy:
\begin{align*}
p \approx -\frac{1}{2} \log 2\pi + \left(N + \frac{1}{2}\right) \log N \\
         - \frac{N+1}{2} \left[ 2\log(\frac{N}{2}) + \frac{2z}{N} - \frac{2z}{N} \right] \\
         - z \left[\frac{2z}{N} + \frac{2z}{N} \right] \\
         - N \log 2 = \\ \\
          -\frac{1}{2} \log 2\pi + \left(N + \frac{1}{2}\right) \log N \\
          - (N+1) \left[ \log(N) - \log(2) \right] \\
          - \frac{4z^2}{N} \\
          - N \log 2 = \\
          -\frac{1}{2} \log 2\pi - \frac{1}{2} \log N + \log 2 - \frac{2z^2}{N} = \\
          -\frac{1}{2} \log 2\pi - \frac{1}{2} \log N + \frac{1}{2} \log 2^{2} - \frac{2z^2}{N} = \\
          -\frac{1}{2} \log \left( \frac{4}{\pi N} \right) - \frac{2z^2}{N}
\end{align*}

Powracając do $p$:
\begin{align*}
    p(z) = \sqrt{\frac{2}{\pi N}} e^{-\frac{2z^2}{N}}
\end{align*}

Wracając do zmiennej $x = (2k - N)a$:
\begin{align*}
    z = k - \frac{N}{2} = \frac{x}{2a} + \frac{N}{2}  - \frac{N}{2} = \frac{x}{2a}
\end{align*}

Musimy uwzględnić nie tylko zmianę zmiennej, ale też fakt, 
że $p(z)dz = p(x)dx$, czyli $p(x) = p(z) \frac{dz}{dx}$:

\begin{align*}
    p(x) = \sqrt{\frac{2}{\pi N}} e^{-\frac{x^2}{2Na^2}} \frac{dz}{dx}  = \\
        = \sqrt{\frac{2}{\pi N}} e^{-\frac{x^2}{2Na^2}} \cdot \frac{1}{2a} = \\
        = \sqrt{\frac{1}{2\pi Na^2}} e^{-\frac{x^2}{2Na^2}} = \\
    \frac{1}{\sqrt{2\pi \sigma^2}} e^{-\frac{(x - \mu)^2}{2\sigma^2}}, 
    ~ \mu = 0, ~ \sigma^2 = Na^2
\end{align*}

To samo z użyciem centralnego twierdzenia granicznego (CTG).
Położenie to suma $N$ niezależnych zmiennych losowych o 
wartościach $\pm a$ z prawdopodobieństwami $1/2$. 
Z CTG wynika, że rozkład sumy tych zmiennych losowych zmierza do rozkładu 
normalnego dla $N \to \infty$. W przypadku gdy składniki sumy są losowane 
z tego samego rozkładu - dwumianiowego, parametry granicznego rozkładu to:
\begin{align*}
    \mu = N \langle x \rangle = N(2<k> - N)a  = N ( 2\frac{1}{2}N - \frac{1}{2})a = 0 \\
    \sigma^2 = N \sigma_{x}^{2} = N4a \sigma_{k} = Na^2
\end{align*}

Ostatni krok: zamiana z liczby kroków $N$ na czas $t$. 
Jeśli skoki wykonywane są co $\delta t$, to:
\begin{align*}
    N = \frac{t}{\delta t} \\
    \sigma^2 = Na^2 = \frac{t}{\delta t} a^2 = 2Dt, \quad D = \frac{a^2}{2\delta t}
\end{align*}

Ostatecznie postać rozkładu prawdopodobieństa w granicy ciągłej to:
\begin{align*}
    p(x|t) = \frac{1}{\sqrt{4\pi Dt}} e^{-\frac{x^2}{4Dt}}
\end{align*}

\newpage
\end{solution}
%%%%%%%%%%%%%%%%%%%%%%%%%%%%%%%%%%%%%%%%%%%%%%%%%%%%%%%%
%%%%%%%%%%%%%%%%%%%%%%%%%%%%%%%%%%%%%%%%%%%%%%%%%%%%%%%%
\zadanie[Równanie dyfuzji]

Proszę rozwiązać jednowymiarowe równanie dyfuzji:
\begin{align*}
    \frac{\partial c}{\partial t} = D \frac{\partial^2 c}{\partial x^2}
\end{align*}

z warunkiem początkowym $c(x, 0) = \delta(x)$ i brzegowym 
$c(x \to \pm \infty, t) = 0$.

{\bf Wskazówki:}

\begin{solution}

Rozwiążemy to równianie w przestrzeni odwrotnej do  
której przejedziemy korzystając z transformacji Fouriera:
\begin{align*}
    \hat{c}(k, t) = \int_{-\infty}^{\infty} c(x, t) e^{-ikx} dx \\
    c(x, t) =  \int_{-\infty}^{\infty} \hat{c}(k, t) e^{ikx} dk
\end{align*}

W stawiamy $c(x, t)$ wyrażone przez odwrotną transformację Fouriera do równania:
\begin{align*}
    \frac{\partial \hat{c}}{\partial t} \int_{-\infty}^{\infty} \hat{c}(k,t) e^{ikx} dk = 
     D\frac{\partial^{2}}{\partial x^2}  \int_{-\infty}^{\infty} \hat{c}(k,t) e^{ikx} dk \\
    \int_{-\infty}^{\infty} \frac{\partial \hat{c}(k,t)}{\partial t} e^{ikx} dk = 
     D \int_{-\infty}^{\infty} \hat{c}(k,t) \frac{\partial^2}{\partial x^2} e^{ikx}dk \\
    \int_{-\infty}^{\infty} \frac{\partial \hat{c}(k,t)}{\partial t} e^{ikx} dk = 
     D \int_{-\infty}^{\infty} \hat{c}(k,t) (-k^2) e^{ikx}dk \\
\end{align*}

Porównujemy wyrażenia podcałkowe:
\begin{align*}
    \frac{\partial \hat{c}(k,t)}{\partial t} = -Dk^2 \hat{c}(k,t)
\end{align*}

i mamy równanie różniczkowe pierwszego rzędu dla $\hat{c}(k,t)$, 
którego rozwiązaniem jest:
\begin{align*}
    \hat{c}(k,t) = \hat{c}(k,0) e^{-Dk^2 t}
\end{align*}

Warunek początkowy $c(x, 0) = \delta(x)$ w przestrzeni odwrotnej to $\hat{c}(k,0) = 1$,
więc ostatecznie:
\begin{align*}
    \hat{c}(k,t) = e^{-Dk^2 t}
\end{align*}

Wracamy do przestrzeni rzeczywistej:
\begin{align*}
    c(x,t) = \int_{-\infty}^{\infty} e^{-Dk^2 t} e^{ikx} dk = 
    \int_{-\infty}^{\infty} e^{-Dk^2 t + ikx} dk = 
    \frac{1}{\sqrt{4\pi Dt}} e^{-\frac{x^2}{4Dt}}
\end{align*}

Ostatecznie:

\begin{align*}
    c(x, t) = \frac{1}{\sqrt{4\pi Dt}} e^{-\frac{x^2}{4Dt}}
\end{align*}

\newpage
\end{solution}

%%%%%%%%%%%%%%%%%%%%%%%%%%%%%%%%%%%%%%%%%%%%%%%%%%%%%%%%%
%%%%%%%%%%%%%%%%%%%%%%%%%%%%%%%%%%%%%%%%%%%%%%%%%%%%%%%%%
\zadanie[Samochody i ciężarówki]

W gazetach napisano, że średnio na cztery ciężarówki jadące po drodze za trzema 
jedzie samochód osobowy, podczas gdy wśród pięciu samochodów osobowych 
jest średnio tylko jeden, za którym jedzie ciężarówka. 
Jaka część pojazdów na drogach to ciężarówki?

\vspace{1cm}
{\bf Rozwiązanie:}
\vspace{1cm}

Zagadnienie rozwiążemy korzystając z łańcucha Markowa, czyli modelując 
problem za pomocą stanów $C$ (ciężarówka) i $S$ (samochód osobowy):
\begin{align*}
    P = \begin{bmatrix}
        P_C \\
        P_S
    \end{bmatrix}
\end{align*}


Kolejne samochody jadące za sobą tworzą sekwencję stanów, a 
prawdopodobieństwa przejścia między stanami można odczytać z treści zadania:
\begin{align*}
    p(C \to C) = \frac{1}{4}, \quad p(C \to S) = \frac{3}{4}, \quad p(S \to C) = \frac{1}{5}, \quad p(S \to S) = \frac{4}{5}
\end{align*}

Przejścia między stanami można zapisać w postaci macierzy:
\begin{align*}
    \hat{p} = \begin{bmatrix}
        p(C \to C) & p(S \to C) \\
        p(C \to S) & p(S \to S)
    \end{bmatrix} = 
    \begin{bmatrix}
        \frac{1}{4} & \frac{1}{5} \\
        \frac{3}{4} & \frac{4}{5}
    \end{bmatrix}
\end{align*}

Ewolucja rozkładu pojazdów to $P(t) = \hat{p}P(0)$.
Interesuje nas rozkład stacjonarny, czyli taki, który nie zmienia się w czasie:
\begin{align*}
    P^{eq} = \hat{p} P^{eq} = \begin{bmatrix}
        \frac{1}{4} & \frac{1}{5} \\
        \frac{3}{4} & \frac{4}{5}
    \end{bmatrix} \begin{bmatrix}
        P_C^{eq} \\
        P_S^{eq}
    \end{bmatrix} 
\end{align*}

Do warunku stacjinarności dodajemy warunek normalizacji $P_C^{eq} + P_S^{eq} = 1$.
Otrzymujemy układ równań:
\begin{align*}
    P_C^{eq} = \frac{1}{4} P_C^{eq} + \frac{1}{5} P_S^{eq} \\
    P_S^{eq} = \frac{3}{4} P_C^{eq} + \frac{4}{5} P_S^{eq} \\
    P_C^{eq} + P_S^{eq} = 1
\end{align*}

Wstawiamy warunek normalizacji do pierwszego równania:
\begin{align*}
    P_C^{eq} = \frac{1}{4} P_C^{eq} + \frac{1}{5} (1 - P_C^{eq}) \\
    P_C^{eq} = \frac{1}{4} P_C^{eq} + \frac{1}{5} - \frac{1}{5} P_C^{eq} \\
    P_C^{eq} = \frac{1}{5} + \frac{1}{20} P_C^{eq} \\
    \frac{19}{20} P_C^{eq} = \frac{1}{5} \\
    P_C^{eq} = \frac{1}{5} \cdot \frac{20}{19} = \frac{4}{19}
\end{align*}

Stacjonarna wartość spotkania ciężarówki to $P_C^{eq} = \frac{4}{19} \simeq 0.2$, 
Czas powrotu ciężarówki: $T_C = \frac{1}{P_C^{eq}} = \frac{19}{4} \simeq 5$, 
czyli średnio trzeba odczekać pięć pojazdów.

%%%%%%%%%%%%%%%%%%%%%%%%%%%%%%%%%%%%%%%%%%%%%%%%%%%%%%%%%%
\newpage
%%%%%%%%%%%%%%%%%%%%%%%%%%%%%%%%%%%%%%%%%%%%%%%%%%%%%%%%%%
\zadanie[Błądzenie przypadkowe z dryfem i ścianką]

Rozważmy błądzenie przypadkowe z prawdopodobieństwem przejścia 
$p(0 \to 1) = 1$, $p(i \to i+1) = p < 0.5$, $p(i \to i-1) = 1-p > 0.5$. 
Znaleźć rozkład równowagowy i odpowiednie równanie ciągłe.

\vspace{1cm}
{\bf Rozwiązanie:}
\vspace{1cm}

Poszukujemy rozkładu stacjonarnego, czyli takiego, który nie zmienia się w czasie.
W stanie stacjonarnym prawdopodobieństwo prawdopodobieństwa przejść "w tę i z powrotem"
są takie same, czyli:
\begin{align*}
    P^{eq}(n) p(n \to m) = P^{eq}(m) p(m \to n)
\end{align*}

Dla stanu początkowego i następnych mamy:
\begin{align*}
    P^{eq}(0) p(0 \to 1) = P^{eq}(1) p(1 \to 0) \\
    P^{eq}(i) p(i \to i+1) = P^{eq}(i+1) p(i+1 \to i) 
\end{align*}

Wyrażamy $P^{eq}(i)$ przez $P^{eq}(0)$ i pamiętamy, że 
$P(0 \to 1) = 1$, $P(i \to i+1) = p$, $P(i \to i-1) = 1-p$:
\begin{align*}
    P^{eq}(1) = P^{eq}(0) \frac{p(0 \to 1)}{p(1 \to 0)} = P^{eq}(0) \frac{1}{1-p} \\
    P^{eq}(2) = P^{eq}(1) \frac{p(1 \to 2)}{p(2 \to 1)} = P^{eq}(1) \frac{p}{1-p} = 
    P^{eq}(0) p^{1}\left(\frac{1}{1-p}\right)^2 \\
    P^{eq}(n) = P^{eq}(0) p^{n-1}\left(\frac{1}{1-p}\right)^n = 
    P^{eq}(0) \left(\frac{p}{1-p}\right)^{n} \frac{1}{p}
\end{align*}

Współczynnik $\frac{p}{1-p}$ jest mniejszy od jedności, 
więc szereg jest zbieżny i możemy znaleźć $P^{eq}(0)$ z warunku normalizacji.
Sumując szereg trzeba pamiętać, że wyrazy 0 i 1 mają inną postać niż pozostałe
, więc sumujemy od 2 do nieskończoności, a potem dodajemy $P^{eq}(0)$ i $P^{eq}(1)$:


\begin{align*}
    \sum_{n=1}^{\infty} P^{eq}(n) = 1 \\
    P^{eq}(0) + \sum_{n=1}^{\infty} P^{eq}(n) = 1 \\
    P^{eq}(0) + \frac{1}{p}\sum_{n=1}^{\infty} P^{eq}(0) \left(\frac{p}{1-p}\right)^{n} = 1\\
    P^{eq}(0) + \frac{1}{p} P^{eq}(0) \sum_{n=1}^{\infty} \left(\frac{p}{1-p}\right)^{n} = 1\\
    P^{eq}(0) \left(1 + \frac{1}{p} \cdot \frac{\frac{p}{1-p}}{1 - \frac{p}{1-p}}\right) = 1 \\
    P^{eq}(0) \left(1 + \frac{1}{p} \cdot \frac{p}{1-2p}\right) = 1 \\
    P^{eq}(0) \left(1 + \frac{1}{1-2p}\right) = P^{eq}(0) \frac{1-2p + 1}{1-2p} = P^{eq}(0) \frac{2(1-p)}{1-2p} = 1 \\
    P^{eq}(0) = \frac{1-2p}{2(1-p)}
\end{align*}

Stąd $P^{eq}(n)$:
\begin{align*}
    P^{eq}(n) = P^{eq}(0) \left(\frac{p}{1-p}\right)^{n} \frac{1}{p} = 
    \frac{1-2p}{2(1-p)} \left(\frac{p}{1-p}\right)^{n-1} = \\
    = \frac{1-2p}{2} \left(\frac{p}{1-p}\right)^{n},~ n \geq 1
\end{align*}

WAriant ciągły. Zaczynamy od równania master ewolucji rozkładu prawdopodobieństwa:
\begin{align*}
    P(i, t + \delta t) =  P(i-1, t) p(i-1 \to i) + P(i+1, t) p(i+1 \to i) = \\
    = P(i-1, t) p + P(i+1, t) (1-p)
\end{align*}


Stany $i$ identyfikujemy z punktami na osi $x$, czyli $i \to x$, a odległość między 
punktami to $\delta x$. Zakładamy, że prawdopodobieństwo przejścia maleje liniowo z 
odległością między punktami: 
\begin{align*}
    p(i \to i+1) = p = \frac{1}{2} - \alpha \delta x 
\end{align*}

Wstawiamy do równania ewolucji:
\begin{align*}
    P(x, t + \delta t) =  P(x - \delta x, t) \left(\frac{1}{2} - \alpha \delta x\right) + P(x + \delta x, t) \left(\frac{1}{2} + \alpha \delta x\right)
\end{align*}

Dązymy do uzyskania pochodnej po czasie, więc odejmujemy $P(x, t)$ 
i dzielimy przez $\delta t$:
\begin{align*}
    \frac{P(x, t + \delta t) - P(x, t)}{\delta t} =  \\
    = \frac{1}{2} \left[ \frac{P(x - \delta x, t) - 2P(x, t) + P(x + \delta x, t)}{\delta t} \right] + 
    \alpha \delta x \left[ \frac{P(x + \delta x, t) - P(x - \delta x, t)}{\delta t} \right]
\end{align*}

Wiążemy $\delta t$ i $\delta x$ wprowadzając skończone  współczynniki 
dyfuzji $D = \frac{(\delta x)^2}{2\delta t}$ 
i  dryfu $U = \alpha (\delta x)^2 / \delta t$:
\begin{align*}
    \frac{P(x, t + \delta t) - P(x, t)}{\delta t} =  \\
    = D \frac{P(x - \delta x, t) - 2P(x, t) + P(x + \delta x, t)}{(\delta x)^2} +
    U \frac{P(x + \delta x, t) - P(x - \delta x, t)}{\delta x}
\end{align*}

W granicy $\delta t \to 0$ i  $\delta x \to 0$ otrzymujemy równianie adwekcji-dyfuzji:
\begin{align*}
    \frac{\partial P}{\partial t} = D \frac{\partial^2 P}{\partial x^2} + U \frac{\partial P}{\partial x}
\end{align*}

Poszukujemy rozwiązania równowagowego, czyli takiego, które nie zmienia się w czasie:
\begin{align*}
    0 = D \frac{\partial^2 P^{eq}}{\partial x^2} + U \frac{\partial P^{eq}}{\partial x} = \\
    \frac{\partial }{\partial x}\left(D \frac{\partial P^{eq}}{\partial x} + U P^{eq}\right) = 0 \rightarrow \\
    D \frac{\partial P^{eq}}{\partial x} + U P^{eq} = C \\
    \frac{\partial P^{eq}}{\partial x} = -\frac{U}{D} P^{eq} + \frac{C}{D}
\end{align*}

Mamy równianie niejednorodne, którego rozwiązaniem jest suma rozwiązania jednorodnego i
szczególnego. Rozwiązanie jednorodne to $P^{eq}_{h} = e^{-\frac{Ux}{D}}$, 
a rozwiązanie szczególne to $P^{eq}_{p} = \frac{C}{U}$, czyli jakaś stała. 
Ostatecznie dostajemy:
\begin{align*}
    P^{eq}(x) = A e^{-\frac{Ux}{D}} + B
\end{align*}

Stałe $A$ i $B$ można znaleźć z warunków brzegowych:
\begin{align*}
    P^{eq}(x \to 0) = 1 \rightarrow A + B = 1 \\
    P^{eq}(x \to \infty) = 0 \rightarrow B = 0 \to A = 1
\end{align*}

Finalnie:
\begin{align*}
    P^{eq}(x) = e^{-\frac{Ux}{D}}
\end{align*}


\end{document}
