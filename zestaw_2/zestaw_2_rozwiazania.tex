\documentclass[11pt,a4paper]{article}

\usepackage[polish]{babel}
\usepackage[utf8]{inputenc}
\usepackage{polski}
\usepackage[T1]{fontenc}
\usepackage{indentfirst}
\usepackage{wrapfig}    % for wrapping figures, tables
\usepackage{isotope}
\usepackage{hyperref}
\usepackage{amsmath}
\usepackage{amssymb}
\usepackage{bm}
\usepackage{gensymb}
\usepackage{epsfig}
\usepackage{graphics}
\usepackage[shortlabels]{enumitem}
\usepackage{xspace}
\xspaceaddexceptions{[]\{\}}

%
%
%fixpagesize
\pagestyle{empty}
\addtolength{\textwidth}{4cm}
\addtolength{\textheight}{6cm}
\addtolength{\evensidemargin}{-3cm}
\addtolength{\oddsidemargin}{-2cm}
\addtolength{\topmargin}{-3cm}
\parindent=0cm

%
%
% Changes figure placing algorithm
\renewcommand{\topfraction}{1}       % maximal fraction of a page allowed for figures
\renewcommand{\textfraction}{0.15}   % minimal number of text for figure-text shared pages
\renewcommand{\floatpagefraction}{0.95} % if two above does not help, this could do the job 
                                        % must be: floatpagefraction < topfraction !!!!
\renewcommand{\textfraction}{0} % minimum fraction of page, which must be  devoted to text
\renewcommand{\topfraction}{1}  % maximum fraction at top, which can be occupied whit floats
\setcounter{totalnumber}{400}   % increase the number of floats for one page
\setcounter{topnumber}{200}     % at all/top/bottom.
\setcounter{bottomnumber}{200}  %

%
%
%small distance in list/item/enum for enumitem package
\setlist[itemize,enumerate]{topsep=0em}
\setlist{noitemsep}

%Nuclear notations: usage \nucl{235}{92}{U}. Math mode optional
\newcommand{\nucl}[3]{\ensuremath{
  \phantom{\ensuremath{^{#1}_{#2}}}
  \llap{\ensuremath{^{#1}}}
  \llap{\ensuremath{_{\rule{0pt}{.75em}#2}}}
  \mbox{#3} } 
} 

%print zadanie #
\newcounter{zadanie}\newcommand{\zadanie}[1][]{\addtocounter{zadanie}{1} ~\\  {\bf \emph{Zadanie \arabic{zadanie} #1 }} \\}
\newcounter{zaddom}\newcommand{\zaddom}[1][]{\addtocounter{zaddom}{1} ~\\  {\bf \emph{Zadanie domowe \arabic{zaddom} #1 }} \\}

%dynamic solutions hiding
\usepackage{comment}

\newif\ifshowsolutions

%\showsolutionstrue   % ← show solutions
% \showsolutionsfalse % ← hide solutions

\def\showsolutions{}

\newif\ifshowsolutions
\ifdefined\showsolutions
  \showsolutionstrue
\else
  \showsolutionsfalse
\fi

\ifshowsolutions
  \newenvironment{solution}{\vspace{1cm} \par\textbf{Rozwiązanie.} \vspace{1cm}}{}
\else
  \excludecomment{solution}
\fi
%%%%%%%%%%%%%%%%%%%%%%%%%%%%%%%%%%%%%%%%%%

\begin{document}         

%%%%%%%%%%%%%%%%%%%%%%%%%%%%%%%%%%%%%%%%%%%%%%%%%%%%%
\begin{centering}
  {\bf {\Large
  Termodynamika i Fizyka Statystyczna R 
  Zestaw II: Rachunek prawdopodobieństwa - kontynuacja
  }} \\  
\end{centering}

\vspace{1cm}
%%%%%%%%%%%%%%%%%%%%%%%%%%%%%%%%%%%%%%%%%%%%%%%%%%%%%%%
\zadanie[Jednowymiarowe błądzenie przypadkowe]

Proszę rozważyć jednowymiarowe błądzenie przypadkowe po sieci o kroku $a$ 
i znaleźć:
\begin{itemize}
    \item rozkład prawdopodobieństwa położenia cząstki po $N$ skokach:
          $p(x|N)$, gdzie $x = Na$ 
    \item postać rozkładu w granicy $a \to 0$, $N \to \infty$ i 
          odpowiedniej korelacji między $a$ i $N$.
\end{itemize}

{\bf Wskazówki:} 
\begin{itemize}
\item użyć zmiennej $z = k - N/2$, gdzie $k$ to liczba skoków na prawo, 
\item użyć przybliżenia Stirlinga dla silni
\end{itemize}


\begin{solution}

Położenie to $x = [k - (N - k)]a = (2k - N)a$, gdzie $k$ to liczba skoków na prawo. 
Prawdopodobieństwo znalezienia cząstki w położeniu $x$ po $N$ skokach to
prawdopodobieństwo, sekwencji $k$ skoków na prawo i $N-k$ skoków na lewo,
pomnożone przez liczbę możliwych sekwencji, czyli:
\begin{align*}
    p(x|N) &= \frac{N!}{k!(N-k)!} \cdot \frac{1}{2^N} 
\end{align*}

Zsymetryzujmy wzór, wprowadzając zmienną $z = k - N/2$, 
czyli liczbę skoków na prawo ponad średnią. Wtedy:

\begin{align*}
    p(z) &= \frac{N!}{(N/2 + z)!(N/2 - z)!} \cdot \frac{1}{2^N}
\end{align*}

Przybliżenie Stirlinga dla silni:
\begin{align*}
    \log n! = \frac{1}{2} \log 2\pi + \left(n + \frac{1}{2}\right) \log n - n + \ldots
\end{align*}

Aplikujemy do wzoru na $p(z)$:
\begin{align*}
    \log p \simeq \\
        \frac{1}{2} \log 2\pi + \left(N + \frac{1}{2}\right) \log N - N - \\
       - \left[\frac{1}{2} \log 2\pi + \left(\frac{N}{2} + z + \frac{1}{2}\right) \log \left(\frac{N}{2} + z\right)  - \frac{N}{2} - z\right]  \\
       - \left[\frac{1}{2} \log 2\pi + \left(\frac{N}{2} - z + \frac{1}{2}\right) \log \left(\frac{N}{2} - z\right)  - \frac{N}{2} + z\right]  \\
       - N \log 2 
\end{align*}


Zbierając:

\begin{align*}
    \log p = -\frac{1}{2} \log 2\pi + \left(N + \frac{1}{2}\right) \log N - \left(N + \frac{1}{2}\right) \log \left[\left(\frac{N}{2} + z\right) \left(\frac{N}{2} - z\right) \right] - z \log \frac{N/2 + z}{N/2 - z} - N \log 2
\end{align*}

Wyrażenia typu $\log(\frac{N}{2} + z)$ można przybliżyć dla $N \gg z$ jako:
\begin{align*}
    \log(\frac{N}{2} + z) = \log \frac{N}{2} + \log(1 + \frac{2z}{N}) \approx \log \frac{N}{2} + \frac{2z}{N}
\end{align*}


Używając tego przybliżenia, dostajemy:
\begin{align*}
p \approx -\frac{1}{2} \log 2\pi + \left(N + \frac{1}{2}\right) \log N \\
         - \frac{N+1}{2} \left[ 2\log(\frac{N}{2}) + \frac{2z}{N} - \frac{2z}{N} \right] \\
         - z \left[\frac{2z}{N} + \frac{2z}{N} \right] \\
         - N \log 2 = \\ \\
          -\frac{1}{2} \log 2\pi + \left(N + \frac{1}{2}\right) \log N \\
          - (N+1) \left[ \log(N) - \log(2) \right] \\
          - \frac{4z^2}{N} \\
          - N \log 2 = \\
          -\frac{1}{2} \log 2\pi - \frac{1}{2} \log N + \log 2 - \frac{2z^2}{N} = \\
          -\frac{1}{2} \log 2\pi - \frac{1}{2} \log N + \frac{1}{2} \log 2^{2} - \frac{2z^2}{N} = \\
          -\frac{1}{2} \log \left( \frac{4}{\pi N} \right) - \frac{2z^2}{N}
\end{align*}

Powracając do $p$:
\begin{align*}
    p(z) = \sqrt{\frac{2}{\pi N}} e^{-\frac{2z^2}{N}}
\end{align*}

Wracając do zmiennej $x = (2k - N)a$:
\begin{align*}
    z = k - \frac{N}{2} = \frac{x}{2a} + \frac{N}{2}  - \frac{N}{2} = \frac{x}{2a}
\end{align*}

Musimy uwzględnić nie tylko zmianę zmiennej, ale też fakt, 
że $p(z)dz = p(x)dx$, czyli $p(x) = p(z) \frac{dz}{dx}$:

\begin{align*}
    p(x) = \sqrt{\frac{2}{\pi N}} e^{-\frac{x^2}{2Na^2}} \frac{dz}{dx}  = \\
        = \sqrt{\frac{2}{\pi N}} e^{-\frac{x^2}{2Na^2}} \cdot \frac{1}{2a} = \\
        = \sqrt{\frac{1}{2\pi Na^2}} e^{-\frac{x^2}{2Na^2}} = \\
    \frac{1}{\sqrt{2\pi \sigma^2}} e^{-\frac{(x - \mu)^2}{2\sigma^2}}, 
    ~ \mu = 0, ~ \sigma^2 = Na^2
\end{align*}

To samo z użyciem centralnego twierdzenia granicznego (CTG).
Położenie to suma $N$ niezależnych zmiennych losowych o 
wartościach $\pm a$ z prawdopodobieństwami $1/2$. 
Z CTG wynika, że rozkład sumy tych zmiennych losowych zmierza do rozkładu 
normalnego dla $N \to \infty$. W przypadku gdy składniki sumy są losowane 
z tego samego rozkładu - dwumianiowego, parametry granicznego rozkładu to:
\begin{align*}
    \mu = N \langle x \rangle = N(2<k> - N)a  = N ( 2\frac{1}{2}N - \frac{1}{2})a = 0 \\
    \sigma^2 = N \sigma_{x}^{2} = N4a \sigma_{k} = Na^2
\end{align*}

Ostatni krok: zamiana z liczby kroków $N$ na czas $t$. 
Jeśli skoki wykonywane są co $\delta t$, to:
\begin{align*}
    N = \frac{t}{\delta t} \\
    \sigma^2 = Na^2 = \frac{t}{\delta t} a^2 = 2Dt, \quad D = \frac{a^2}{2\delta t}
\end{align*}

Ostatecznie postać rozkładu prawdopodobieństa w granicy ciągłej to:
\begin{align*}
    p(x|t) = \frac{1}{\sqrt{4\pi Dt}} e^{-\frac{x^2}{4Dt}}
\end{align*}

\newpage
\end{solution}
%%%%%%%%%%%%%%%%%%%%%%%%%%%%%%%%%%%%%%%%%%%%%%%%%%%%%%%%
%%%%%%%%%%%%%%%%%%%%%%%%%%%%%%%%%%%%%%%%%%%%%%%%%%%%%%%%
\zadanie[Równanie dyfuzji]

Na wykładzie wypisaliśmy równanie Master dla jednowymiarowego 
błądzenia przypadkowego i przeszliśmy do granicy ciągłej, 
co doprowadziło nas do jednowymiarowego równania dyfuzji:
\[
    \frac{\partial c}{\partial t} = D \frac{\partial^2 c}{\partial x^2}
\]

Chciałbym, żebyście znaleźli jego rozwiązanie dla warunku początkowego $c(x, 0) = \delta(x)$, otrzymując Gaussa:
\[
    p = \frac{1}{\sqrt{4\pi Dt}} e^{-\frac{x^2}{4Dt}}
\]

To można robić na kilka sposobów, np. metodą skalowania lub przez transformację Fouriera. Skomentujcie, że finalnie dostajemy to samo co w zadaniu 1.

%%%%%%%%%%%%%%%%%%%%%%%%%%%%%%%%%%%%%%%%%%%%%%%%%%%%%%%%%
\newpage
%%%%%%%%%%%%%%%%%%%%%%%%%%%%%%%%%%%%%%%%%%%%%%%%%%%%%%%%%
\zadanie[Samochody i ciężarówki]

Wiadomo, że średnio na cztery ciężarówki jadące po drodze za trzema 
jedzie samochód osobowy, podczas gdy wśród pięciu samochodów osobowych 
jest średnio tylko jeden, za którym jedzie ciężarówka. Jaka część pojazdów 
na drogach to ciężarówki?

\vspace{1cm}
{\bf Rozwiązanie:}
\vspace{1cm}


Modelowanie problemu za pomocą stanów $C$ (ciężarówka) i $S$ (samochód osobowy). Przejścia między stanami:
\[
    p(C \to C) = \frac{1}{4}, \quad p(C \to S) = \frac{3}{4}, \quad p(S \to C) = \frac{1}{5}, \quad p(S \to S) = \frac{4}{5}
\]

Macierz przejść:
\[
    p = \begin{bmatrix}
        \frac{1}{4} & \frac{3}{4} \\
        \frac{1}{5} & \frac{4}{5}
    \end{bmatrix}
\]

Rozwiązanie stacjonarne:
\[
    P_{C}^{eq} = \frac{4}{19}, \quad P_{S}^{eq} = \frac{15}{19}
\]

Czas powrotu ciężarówki: $T_C = \frac{1}{P_C^{eq}} = \frac{19}{4}$, czyli średnio trzeba odczekać pięć pojazdów.

%%%%%%%%%%%%%%%%%%%%%%%%%%%%%%%%%%%%%%%%%%%%%%%%%%%%%%%%%%
\newpage
%%%%%%%%%%%%%%%%%%%%%%%%%%%%%%%%%%%%%%%%%%%%%%%%%%%%%%%%%%
\zadanie[Błądzenie przypadkowe z dryfem i ścianką]

Rozważmy błądzenie przypadkowe z prawdopodobieństwem przejścia 
$p(0 \to 1) = 1$, $p(i \to i+1) = p < 0.5$, $p(i \to i-1) = 1-p > 0.5$. 
Znaleźć rozkład równowagowy i odpowiednie równanie ciągłe.

\vspace{1cm}
{\bf Rozwiązanie:}
\vspace{1cm}

Rozkład równowagowy:
\[
    P^{eq}(n) = \frac{1-2p}{2(1-p)} \left(\frac{p}{1-p}\right)^{n-1}, \quad n \geq 1
\]

Równanie ciągłe:
\[
    \frac{\partial P}{\partial t} = D \frac{\partial^2 P}{\partial x^2} + U \frac{\partial P}{\partial x}
\]

Rozwiązanie równowagowe:
\[
    P^{eq} = e^{-\frac{Ux}{D}}
\]

\end{document}
